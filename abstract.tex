%%%%%%%%%%%%%%%%%%%%%%%%%%%%%%%%%%%%%%%%%%%%%%%%%%%%%%%%%%%%
\documentclass[11pt,            % Schriftgröße {{{
               a4paper,         % A4
               oneside,         % Einseitig
               DIV12,           % Papiergröße
             % DIV15,           % Größer
             % draft,           % Entwurf
               fleqn,           % Linksbündige Gleichungen
             % headsepline,     % Trennlinie oben
             % footsepline,     % -""-       unten
               smallheadings,   % Kleine Überschriften
             % pointlessnumbers,% Keine Punkte
               halfparskip,     % Halbe Zeile Absatz statt Einzug
               nochapterprefix, % Kein "Kapitel"
             % bibtotoc         % "Literatur" im TOC  oder
               bibtotocnumbered,% -""-, nummeriert
             % idxtotoc,        % Index im TOC
              ]{scrartcl} %%% }}}
%%%%%%%%%%%%%%%%%%%%%%%%%%%%%%%%%%%%%%%%%%%%%%%%%%%%%%%%%%%%

%%%%%%%%%%%%%%%%%%%%%%%%%%%%%%%%%%%%%%%%%%%%%%%%%%%%%%%%%%%%
%%% Pakete {{{
\usepackage[utf8]{inputenc}         % Umlaute etc.
\usepackage[T1]{fontenc}            % T1-kodierte Fonts
\usepackage{ae,aecompl}             % Kodierung für PDF
\usepackage{ngerman}                % Deutsche Trennungen,
                                    % dt. Begriffe
\usepackage{setspace}               % Single- oder Onehalfspacing
\setcounter{tocdepth}{4}            % 4 Hirarchien im Inhaltsv.
\usepackage{times}                  % Times als Schrift
\usepackage{amsmath,amssymb,amstext}% Mathematische Symbole
\usepackage{exscale}                % Skalierung von Summen-c und Int.-zeichen
\usepackage{url}                    % Darstellung von URLs
\usepackage{calc}

%%% Optional, je nach Dokument
% \usepackage{listings}             % Quelltext-Listings
% \usepackage{units}                % Technische Units
% \usepackage{psfrag}               % Ersetzts PS-Schriften
  \usepackage{color}                % Farben in LaTeX
% \usepackage{floatflt}             % Textumflossene Bilder...
% \usepackage{picins}               % Textumflossene Bilder
  \usepackage{textcomp}             % Spezielle Zeichen
  \usepackage{gensymb}              % Spezielle Zeichen
% \usepackage{eurosym}              % Euro-Symbol
% \usepackage{currvita}             % Befehle für CVs
  \usepackage{ifpdf}                % Wird ein PDF erstellt?

%%% Layout
\usepackage{scrpage2}               % KOMA-Überschriften und -Fußzeilen.
%%% }}}
%%%%%%%%%%%%%%%%%%%%%%%%%%%%%%%%%%%%%%%%%%%%%%%%%%%%%%%%%%%%

%%%%%%%%%%%%%%%%%%%%%%%%%%%%%%%%%%%%%%%%%%%%%%%%%%%%%%%%%%%%
%%% PDF {{{

\ifpdf
  \usepackage[pdftex]{graphicx}
  \DeclareGraphicsExtensions{.pdf}
  \pdfcompresslevel=9
  \usepackage[%
    pdftex=true,
    backref=true,
    colorlinks=true,
    bookmarks=true,
    breaklinks=true,
    linktocpage=true,
    bookmarksopen=false,
    bookmarksnumbered=false,
    pdfpagemode=None
  ]{hyperref}
  \hypersetup{
    pdftitle={},
    pdfauthor={Julius Plenz},
    pdfsubject={},
    pdfcreator={LaTeX2e and pdfLaTeX},
    pdfproducer={},
    pdfkeywords={}
  }
\else
  \usepackage[dvips]{graphicx}
  \DeclareGraphicsExtensions{.eps}
  % \usepackage[%
  %   dvips,
  %   breaklinks=true,
  %   colorlinks=false
  % ]{hyperref}
\fi

%%% }}}
%%%%%%%%%%%%%%%%%%%%%%%%%%%%%%%%%%%%%%%%%%%%%%%%%%%%%%%%%%%%

%%%%%%%%%%%%%%%%%%%%%%%%%%%%%%%%%%%%%%%%%%%%%%%%%%%%%%%%%%%%
%%% Eigene Funktionen {{{
%%% Beispiel:  \bild{200pt}{foo}{That's a foo\ldots}
\newcommand{\bild}[3]{
  \begin{figure}
    \includegraphics[width=#1, keepaspectratio=true]{#2}
    \caption{#3}
    \label{#2}
  \end{figure}
}

%% \floatimg{filename}{caption+label}{r/l}{8cm}
\newcommand{\floatimg}[4]{
  \piccaption{#2}
  \parpic[#3]{\includegraphics[width=#4]{#1}}
}
%%% }}}
%%%%%%%%%%%%%%%%%%%%%%%%%%%%%%%%%%%%%%%%%%%%%%%%%%%%%%%%%%%%

%%%%%%%%%%%%%%%%%%%%%%%%%%%%%%%%%%%%%%%%%%%%%%%%%%%%%%%%%%%%
%%% Pagestyle {{{
  \pagestyle{scrheadings}
% \pagestyle{fancyhdrs}
% \pagestyle{empty}
%%% }}}
%%%%%%%%%%%%%%%%%%%%%%%%%%%%%%%%%%%%%%%%%%%%%%%%%%%%%%%%%%%%

%%%%%%%%%%%%%%%%%%%%%%%%%%%%%%%%%%%%%%%%%%%%%%%%%%%%%%%%%%%%
%%% Seitenkopf- und -Fußzeilen {{{
 \automark[subsection]{section} % \left- und \rightmark bekommen Inhalt
%%% Oben: Links, Mitte, Rechts
 \ihead[]{}
 \chead[]{}
 \ohead[]{}
%%% Unten: Links, Mitte, Rechts
 \ifoot[]{}
 \cfoot[]{}
 \ofoot[]{}
%%% }}}
%%%%%%%%%%%%%%%%%%%%%%%%%%%%%%%%%%%%%%%%%%%%%%%%%%%%%%%%%%%%

%%%%%%%%%%%%%%%%%%%%%%%%%%%%%%%%%%%%%%%%%%%%%%%%%%%%%%%%%%%%
%%% Sonstiges {{{
% \setlength{\parindent}{17pt}      % Einzug 17pt,
% \setlength{\parskip}{2pt}         % keine Leerzeilen.

% \textwidth      127mm             % Textbreite
% \textheight     235mm             % Texthöhe
% \topmargin     -5mm               % Abstand oben
% \oddsidemargin  7mm               % Abstand Links, onepage

%\onehalfspacing                    % Zeilenabstand: Bei korrektur,
 \singlespacing                     % bei Abgabe

% Punkt- und Komma Abstände bei Tausendern/
% Dezimalzahlen ans deutsche anpassen!
 \mathcode`,="013B
 \mathcode`.="613A

 \setlength{\emergencystretch}{2em} % Notfallsstreckung
 \addtolength{\voffset}{10pt}

% Kommandoänderungen
 \renewcommand{\figurename}{Abb.} % Bildunterschriften: Abb. anstatt Fig.
%\renewcommand*{\cvheadingfont}{\raggedleft\Huge\bfseries} % CV: Überschriften
%%% }}}
%%%%%%%%%%%%%%%%%%%%%%%%%%%%%%%%%%%%%%%%%%%%%%%%%%%%%%%%%%%%

\begin{document}

%%%%%%%%%%%%%%%%%%%%%%%%%%%%%%%%%%%%%%%%%%%%%%%%%%%%%%%%%%%%
%%% Titelseite {{{
%\pagenumbering{none}                % Für die Titelseite: Keine Seitennummern,
%\thispagestyle{empty}               % keine Kopf- und Fußzeilen.
%
%\begin{center}
%
%\end{center}
%\newpage
%
%%% }}}
%%%%%%%%%%%%%%%%%%%%%%%%%%%%%%%%%%%%%%%%%%%%%%%%%%%%%%%%%%%%

%%%%%%%%%%%%%%%%%%%%%%%%%%%%%%%%%%%%%%%%%%%%%%%%%%%%%%%%%%%%
%%% Inhaltsverzeichnis {{{
  \pagenumbering{arabic}            % Arabische Nummerierung
% \pagenumbering{roman}             % Kleine, römische Nummerierung
% \tableofcontents                  % Das Inhaltsverzeichnis
% \listoffigures                    % Verzeichnis aller Abbildungen
% \listoftables                     % Verzeichnis aller Tabellen
% \pagenumbering{arabic}            % ...und wieder Arabisch
% \newpage
%%% }}}
%%%%%%%%%%%%%%%%%%%%%%%%%%%%%%%%%%%%%%%%%%%%%%%%%%%%%%%%%%%%

%%%%%%%%%%%%%%%%%%%%%%%%%%%%%%%%%%%%%%%%%%%%%%%%%%%%%%%%%%%%
%%% Inhalt {{{



\section*{Git-Workshop}

Seit die Linux-Kernel-Entwickler ihr Versionsverwaltungssystem Marke
Eigenbau vorgestellt haben, ist \emph{Git} in aller Munde. Das
Programm ist schnell, zuverlässig und revulotioniert die Art und
Weise, wie Versionsverwaltung genutzt wird.

Der Workshop wird mit einer kleinen Einführung starten: grundlegende
Arbeistsschritte und inwieweit sich \emph{Git} von anderen Systemen
wie beispielsweise \emph{SVN} oder \emph{CVS} unterscheidet.
Vor allem aber wollen wir den Teilnehmern Gelegenheit geben, eigene
Fragen oder Unklarheiten vorzubringen.

\subsection*{Vorkenntnisse}

Wenn Sie an diesem Workshop teilnehmen wollen, sollten Sie nach
Möglichkeit ein wenig Erfahrung mit Versionsverwaltungssystemen
mitbringen. (Wenn Sie z.\,B. schon mal mit \emph{SVN} gearbeitet
haben, ist das eine gute Voraussetzung!)

\subsection*{Vorbereitung}

Sie werden aber mehr von dem Workshop mitnehmen, wenn Sie sich schon
vorher ein wenig mit \emph{Git} beschäftigt haben. Dafür seien Ihnen
die folgenden URLs ans Herz gelegt:

% Gerne mehr hinzufügen. Speziell ein deutsches Tutorial wäre
% hilfreich, ich kenne aber gar keines...
\begin{itemize}
\item Umfangreiche Tipsammlung:
  \url{http://www.gitready.com/}
\item Radiosendung mit Thema Versionskontrollsysteme:\\
  \url{http://chaosradio.ccc.de/cre130.html}
\item Die offizielle \emph{Git}-Webseite:
  \url{http://git-scm.com/}
\item \emph{Git}- und \emph{SVN}-Kommandos im direkten Vergleich:\\
  \url{http://git.or.cz/course/svn.html}
\item Interessante beschreibung der internen Datenstrukturen für den ambitionierten Nutzer:\\
    %\url{http://www.newartisans.com/2008/04/git-from-the-bottom-up.html}
    % ... und meinst Du nicht das PDF? -> 
    % Ja, klar!
    \url{http://ftp.newartisans.com/pub/git.from.bottom.up.pdf}
\end{itemize}

Im Workshopraum sind Rechner vorhanden, die Sie benutzen können. Wenn
Sie allerdings einen Laptop haben, so bringen Sie diesen mit! So haben
Sie später alle verwendeten Dateien und Arbeitsschritte auf ihrem
Rechner und befinden sich in einer gewohnten Arbeitsumgebung.

% b) Weil globale Einstellung die wir vorgenommen haben erhalten
% bleiben. (Das gilt auch für Software)
% -> user.email, etc? Kann mana uch schnell neu einstellen... :P



\subsection*{Inhalt des Workshops und Kontakt}

Um spezielle Arbeitsschritte vorzustellen, sowie den Umgang mit
Problemen (wie zum Beispiel Merge-Konflikten), wollen wir mit allen
Workshop-Teilnehmern gleichzeitig ein Projekt erstellen und parallel
daran arbeiten.

Die Idee ist, eine \emph{Git}-Tip-Sammlung aufzubauen, in der jeder
seine eigenen Tips und Arbeitsgänge einließen lassen kann.
Wenn Sie allerdings eine andere Idee, oder auch Fragen und Anregungen
im Vorfeld haben, so teilen Sie uns diese per E-Mail mit, damit wir
speziell darauf im Workshop eingehen können!

\hfill
\begin{minipage}{5cm}
\begin{center}
Valentin Hänel\\
\url{valentin.haenel@gmx.de}
\end{center}
\end{minipage}
\hfill
\begin{minipage}{5cm}
\begin{center}
Julius Plenz\\
\url{clt-git@plenz.com}
\end{center}
\end{minipage}
\hfill\\






%%% }}}
%%%%%%%%%%%%%%%%%%%%%%%%%%%%%%%%%%%%%%%%%%%%%%%%%%%%%%%%%%%%

%%%%%%%%%%%%%%%%%%%%%%%%%%%%%%%%%%%%%%%%%%%%%%%%%%%%%%%%%%%%
%%% Bibliographieverzeichnis {{{
%\newpage
%\nocite{*}
%\bibliographystyle{plaindin}
%\bibliography{quellen}
%%% }}}
%%%%%%%%%%%%%%%%%%%%%%%%%%%%%%%%%%%%%%%%%%%%%%%%%%%%%%%%%%%%

\end{document}

%%% vim:set fdm=marker:
