%%%%%%%%%%%%%%%%%%%%%%%%%%%%%%%%%%%%%%%%%%%%%%%%%%%%%%%%%%%%
\documentclass[9pt,            % Schriftgröße {{{
               a4paper,         % A4
               landscape,
               halfparskip,
               oneside,         % Einseitig
               DIV94,           % Papiergröße
              ]{scrartcl} %%% }}}
%%%%%%%%%%%%%%%%%%%%%%%%%%%%%%%%%%%%%%%%%%%%%%%%%%%%%%%%%%%%

%%%%%%%%%%%%%%%%%%%%%%%%%%%%%%%%%%%%%%%%%%%%%%%%%%%%%%%%%%%%
%%% Pakete {{{
\usepackage[utf8]{inputenc}         % Umlaute etc.
\usepackage[T1]{fontenc}            % T1-kodierte Fonts
\usepackage{ae,aecompl}             % Kodierung für PDF
\usepackage{ngerman}                % Deutsche Trennungen,
                                    % dt. Begriffe
\usepackage{setspace}               % Single- oder Onehalfspacing
\setcounter{tocdepth}{4}            % 4 Hirarchien im Inhaltsv.
\usepackage{times}                  % Times als Schrift
\usepackage{amsmath,amssymb,amstext}% Mathematische Symbole
\usepackage{exscale}                % Skalierung von Summen-c und Int.-zeichen
\usepackage{url}                    % Darstellung von URLs
\usepackage{calc}

%%% Optional, je nach Dokument
% \usepackage{listings}             % Quelltext-Listings
% \usepackage{units}                % Technische Units
% \usepackage{psfrag}               % Ersetzts PS-Schriften
  \usepackage{color}                % Farben in LaTeX
% \usepackage{floatflt}             % Textumflossene Bilder...
% \usepackage{picins}               % Textumflossene Bilder
  \usepackage{textcomp}             % Spezielle Zeichen
  \usepackage{gensymb}              % Spezielle Zeichen
% \usepackage{eurosym}              % Euro-Symbol
% \usepackage{currvita}             % Befehle für CVs
  \usepackage{ifpdf}                % Wird ein PDF erstellt?

%%% Layout
\usepackage{scrpage2}               % KOMA-Überschriften und -Fußzeilen.
%%% }}}
%%%%%%%%%%%%%%%%%%%%%%%%%%%%%%%%%%%%%%%%%%%%%%%%%%%%%%%%%%%%

%%%%%%%%%%%%%%%%%%%%%%%%%%%%%%%%%%%%%%%%%%%%%%%%%%%%%%%%%%%%
%%% PDF {{{

\ifpdf
  \usepackage[pdftex]{graphicx}
  \DeclareGraphicsExtensions{.pdf}
  \pdfcompresslevel=9
  \usepackage[%
    pdftex=true,
    backref=true,
    colorlinks=true,
    bookmarks=true,
    breaklinks=true,
    linktocpage=true,
    bookmarksopen=false,
    bookmarksnumbered=false,
    pdfpagemode=None
  ]{hyperref}
  \hypersetup{
    pdftitle={},
    pdfauthor={Julius Plenz},
    pdfsubject={},
    pdfcreator={LaTeX2e and pdfLaTeX},
    pdfproducer={},
    pdfkeywords={}
  }
\else
  \usepackage[dvips]{graphicx}
  \DeclareGraphicsExtensions{.eps}
  % \usepackage[%
  %   dvips,
  %   breaklinks=true,
  %   colorlinks=false
  % ]{hyperref}
\fi

%%% }}}
%%%%%%%%%%%%%%%%%%%%%%%%%%%%%%%%%%%%%%%%%%%%%%%%%%%%%%%%%%%%

%%%%%%%%%%%%%%%%%%%%%%%%%%%%%%%%%%%%%%%%%%%%%%%%%%%%%%%%%%%%
%%% Eigene Funktionen {{{
%%% Beispiel:  \bild{200pt}{foo}{That's a foo\ldots}
\newcommand{\bild}[3]{
  \begin{figure}
    \includegraphics[width=#1, keepaspectratio=true]{#2}
    \caption{#3}
    \label{#2}
  \end{figure}
}

%% \floatimg{filename}{caption+label}{r/l}{8cm}
\newcommand{\floatimg}[4]{
  \piccaption{#2}
  \parpic[#3]{\includegraphics[width=#4]{#1}}
}
%%% }}}
%%%%%%%%%%%%%%%%%%%%%%%%%%%%%%%%%%%%%%%%%%%%%%%%%%%%%%%%%%%%

%%%%%%%%%%%%%%%%%%%%%%%%%%%%%%%%%%%%%%%%%%%%%%%%%%%%%%%%%%%%
%%% Pagestyle {{{
% \pagestyle{scrheadings}
% \pagestyle{fancyhdrs}
  \pagestyle{empty}
%%% }}}
%%%%%%%%%%%%%%%%%%%%%%%%%%%%%%%%%%%%%%%%%%%%%%%%%%%%%%%%%%%%

%%%%%%%%%%%%%%%%%%%%%%%%%%%%%%%%%%%%%%%%%%%%%%%%%%%%%%%%%%%%
%%% Seitenkopf- und -Fußzeilen {{{
 \automark[subsection]{section} % \left- und \rightmark bekommen Inhalt
%%% Oben: Links, Mitte, Rechts
 \ihead[]{}
 \chead[]{}
 \ohead[]{}
%%% Unten: Links, Mitte, Rechts
 \ifoot[]{}
 \cfoot[]{}
 \ofoot[]{}
%%% }}}
%%%%%%%%%%%%%%%%%%%%%%%%%%%%%%%%%%%%%%%%%%%%%%%%%%%%%%%%%%%%

\usepackage{tikz}
\usetikzlibrary{shapes,decorations,shadows}

%%%%%%%%%%%%%%%%%%%%%%%%%%%%%%%%%%%%%%%%%%%%%%%%%%%%%%%%%%%%
%%% Sonstiges {{{
% \setlength{\parindent}{17pt}      % Einzug 17pt,
% \setlength{\parskip}{2pt}         % keine Leerzeilen.

% \textwidth      127mm             % Textbreite
% \textheight     235mm             % Texthöhe
% \topmargin     -5mm               % Abstand oben
% \oddsidemargin  7mm               % Abstand Links, onepage

%\onehalfspacing                    % Zeilenabstand: Bei korrektur,
 \singlespacing                     % bei Abgabe

% Punkt- und Komma Abstände bei Tausendern/
% Dezimalzahlen ans deutsche anpassen!
 \mathcode`,="013B
 \mathcode`.="613A

 \setlength{\emergencystretch}{2em} % Notfallsstreckung
 \addtolength{\voffset}{10pt}

% Kommandoänderungen
 \renewcommand{\figurename}{Abb.} % Bildunterschriften: Abb. anstatt Fig.
%\renewcommand*{\cvheadingfont}{\raggedleft\Huge\bfseries} % CV: Überschriften
%%% }}}
%%%%%%%%%%%%%%%%%%%%%%%%%%%%%%%%%%%%%%%%%%%%%%%%%%%%%%%%%%%%

\begin{document}

%%%%%%%%%%%%%%%%%%%%%%%%%%%%%%%%%%%%%%%%%%%%%%%%%%%%%%%%%%%%
%%% Inhalt {{{


\tikzstyle{bluebox} = [fill=blue!25, draw=blue!50, very thick,font=\fontfamily{phv}\selectfont]
\tikzstyle{repo} = [fill=green!35, draw=black, very thick, font=\fontfamily{phv}\selectfont]
\tikzstyle{mensch} = [ellipse,minimum height=1.3cm, fill=orange!35, draw=orange!85, very thick, font=\fontfamily{phv}\selectfont\Large]

\tikzstyle{push} = [->, orange, line width=.7mm]
\tikzstyle{push-label} = [near start,sloped,above]

\tikzstyle{pull} = [<-, blue, line width=.7mm]
\tikzstyle{pull-label} = [very near start,sloped,above]

\tikzstyle{kommunikation} = [->, black!75, line width=.3mm, dotted]

% Rehner und Repositories
\vspace*{3cm}
\begin{center}
\begin{tikzpicture}
\node [bluebox, minimum width=10cm, minimum height=8cm, rounded corners=7pt,inner sep=.7cm] (deepthought) { } ;
\node [bluebox, rounded corners=2pt] (deepthought-label) at (deepthought.north) {\textbf{git.plenz.com}};
\node [repo, minimum height=2cm] (blessed-repo) at ([yshift=1.5cm] deepthought) {
    \begin{minipage}{7cm}
    \begin{center}
    Blessed \\ Repository
    \end{center}
    \end{minipage}
};
\node [repo, anchor=north west, minimum height=1cm] (tn01-repo) at ([yshift=-1cm] blessed-repo.south west) {
    \begin{minipage}{2cm}
    \begin{center}
    Repository\\\textbf{\texttt{tn01}}
    \end{center}
    \end{minipage}
};
\node [repo, anchor=north east, minimum height=1cm] (tn02-repo) at ([yshift=-1cm] blessed-repo.south east) {
    \begin{minipage}{2cm}
    \begin{center}
    Repository\\\textbf{\texttt{tn02}}
    \end{center}
    \end{minipage}
};

% Die Teilnehmer
\node [mensch] (maintainer) at ([xshift=6cm] blessed-repo.east) { \textbf{Maintainer} };
\node [mensch] (tn02) at ([yshift=-8cm] maintainer.south) { \textbf{Teilnehmer 2} };
\node [mensch] (tn01) at ([xshift=-7cm] tn02.west) { \textbf{Teilnehmer 1} };

% Push
\draw[push] (tn01.160) -- node[push-label]{push} (tn01-repo.south);
\draw[push] (tn02.160) -- node[push-label]{push} (tn02-repo.south);
\draw[push] (maintainer.west) -- node[push-label]{push} (blessed-repo.east);

% Pull
\draw[pull]  (tn01.north) -- node[pull-label]{pull} (blessed-repo.220);
\draw[pull]  (tn02.west) .. controls +(-2,0) and +(0,-3) .. node[pull-label]{pull} (blessed-repo.south);
\draw[pull, dashed]  (maintainer.-165) -- node[pull-label,near start]{pull} (tn02-repo.east);

% Kommunikation
\draw[kommunikation] (tn02.north) -- node[midway,sloped,above]{Pull-Request} (maintainer.south);
\draw[kommunikation,<-] (tn01.-30)
    .. controls +(2,-0.5) and +(-2,-0.5) ..
    node[left=.8cm,midway,below]{Sendet Patch per E-Mail}
    (tn02.190);

\end{tikzpicture}
\end{center}


%%% }}}
%%%%%%%%%%%%%%%%%%%%%%%%%%%%%%%%%%%%%%%%%%%%%%%%%%%%%%%%%%%%

\end{document}

%%% vim:set fdm=marker:
